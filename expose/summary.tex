%
% Sample introduction of your thesis
%

\chapter{Expos\'{e}}
Despite the rapid progress in the developement of quantum computers and the hereby increasingly urgent need for post-quantum cryptographic algorithms, no such algorithms has yet been standardised \cite{Nist}.\\
Our paper will give an overview over some selected lattice-based algorithms and their implementation in respect to their resistance to various side-channel attack techniques.\\
A short introduction to the topic of lattice-based cryptography and its advantages in prospect to quantum computers, %Komma???
as well as the importance of implementations of such algorithms being resistant to side-channel attacks will be given in Section 1 of our paper. Section 2 will explain our notation and give an overview over the mathematic background information needed to understand this paper. This includes introducing the reader to the concept of \textit{(ideal) lattices}, the \textit{Learning with Errors Problem (LWE)} (both described in \cite{cryptoeprint:2012:230}), \textit{Discrete Gaussion Distributions}, the \textit{Ring-LWE Encryption Scheme} and the \textit{BLISS Signature Scheme}. Additionally we will give a short explanation of the side-channel attack terminology used throughout this paper, which will be similar to the one used in \cite{DBLP:conf/crypto/KocherJJ99}, \cite{Kocher2011}, \cite{cryptoeprint:2010:646} and \cite{cryptoeprint:2010:385}. Section 3 will deal the the Ring-LWE Encryption Scheme and will be split in two parts, starting with the description of a masked implementation of this algorithm, including a masked decoder build upon a masked table lookup (as described in \cite{maskedRing}). As \textit{masking} is a technique used to prevent an attacker from gaining intermediate information through side-channels while the algorithm is being executed, the second part of this section will be an evaluation of the proposed implementation in respect to its soundness to first- and second-order side-channel attacks.\\

RingLWE Masking ohne Table: \cite{Reparaz2016} \\
RingLWE Implementation: \cite{Pöppelmann2014} \\
Blinding for RIngLWE: \cite{cryptoeprint:2016:276} \\
Bliss introduction: \cite{cryptoeprint:2013:383} \\
Flush, Gauss and Reload: \cite{cryptoeprint:2016:300} \\