%
% Sample introduction of your thesis
%

\chapter{Expos\'{e}}
Despite the rapid progress in the development of quantum computers and the hereby increasingly urgent need for post-quantum cryptographic algorithms, no such algorithms has yet been standardized \cite{Nist}.\\
Our paper will give an overview over some selected lattice-based algorithms and their implementation in respect to their resistance to various side-channel attack techniques.\\
A short introduction to the topic of lattice-based cryptography and its advantages in prospect to quantum computers, %Komma???
as well as the importance of implementations of such algorithms being resistant to side-channel attacks will be given in Section 1 of our paper.\\ 
Section 2 will explain our notation and give an overview over the mathematic background information needed to understand this paper. This includes introducing the reader to the concept of \textit{(ideal) lattices}, the \textit{learning with errors problem (LWE)} (both described in \cite{cryptoeprint:2012:230}), \textit{Discrete Gaussian Distributions}, the \textit{ring-LWE Encryption Scheme} and the \textit{BLISS Signature Scheme} (as introduced in \cite{cryptoeprint:2013:383} and implemented in \cite{Pöppelmann2014}). Additionally, we will give a short explanation of the side-channel attack terminology used throughout this paper, which will be similar to the one used in \cite{DBLP:conf/crypto/KocherJJ99}, \cite{Kocher2011}, \cite{cryptoeprint:2010:646} and \cite{cryptoeprint:2010:385}.\\ 
Section 3 will deal the ring-LWE encryption scheme and will be split in two parts, starting with the description of a masked implementation of the decryption function, including a masked decoder build upon a masked table lookup (as described in \cite{maskedRing}). As \textit{masking} is a technique used to prevent an attacker from gaining intermediate information through side-channels while the algorithm is being executed, the second part of this section will be an evaluation of the proposed implementation in respect to its soundness to first- and second-order side-channel attacks.\\ 
A different approach to masking of the ring-LWE encryption scheme \cite{Reparaz2016} will be presented in Section 4 of our paper, which will as well be split into a description of the proposed scheme and an evaluation. Furthermore, the second masking scheme will be compared to the first one in respect to efficiency and complexity.\\
Section 5 will discuss the \verb|FLUSH+RELOAD| cache attacks on the Gaussian sampler used in the BLISS signature scheme, as proposed in \cite{cryptoeprint:2016:300}. 
This part will start with a description of a perfect side channel attack on two Gaussian sampling algorithms, namely the cumulative distribution function (CDT sampling) proposed in \cite{cryptoeprint:2010:088} and refined and implemented in \cite{cryptoeprint:2014:254} and rejection sampling \cite{cryptoeprint:2013:383}. This will be followed by an evaluation of the \verb|FLUSH+RELOAD| attacks on a BLISS implementation, which is running on modern CPUs.\\
Furthermore, in Section 6 we will be presenting two measures used for blinding polynomial multiplication and Gaussian sampling \cite{cryptoeprint:2016:276}, which might help against the attacks described in Section 5. As both, polynomial multiplication and Gaussian sampling, are generic operations used in most lattice based cryptosystems, those countermeasures can be used in a much broader way than the masking approaches detailed in Section 3 and 4.\\
Finally, Section 7 will summarize the content of our paper shortly and some conclusions will be drawn.