% 
% EMSEC seminar template, 24.09.2010
% david.oswald@rub.de
% 
% based on
% EMSEC thesis template, 13.09.2010
% 
% Calls all chapters
% Each chapter is contained in a seperate "\input" file 
% Style: doublesided, scrreport, dina4
%
% Benedikt Driessen, 2010
% benedikt.driessen@rub.de
%

% *************************************************************
% ENTER INFORMATION ABOUT AUTHOR, TITLE and TYPE HERE
\newcommand{\thauthor}{Julian Speith, Felix Haarmann}
\newcommand{\thtitle}{Side-Channel Attacks on Implementations of Lattice-Based Cryptosystems}
% *************************************************************

\documentclass[a4paper,12pt,twoside,openany,headsepline,bibliography=totocnumbered]{scrbook}

% Import settings, packages, etc.
\input{settings}

\begin{document}

% Import frontpage
%
% Frontpage ripped from S. Rinne
%

\frontmatter

\begin{titlepage}
 \enlargethispage{3cm}
 \vspace*{-32mm}\hspace*{120mm}
 \includegraphics[scale=1.0]{rub_logo.eps}
 
 \vspace*{12cm}\hspace*{0mm}
 \begin{minipage}[b]{1\linewidth}
  \sffamily
  \hspace{-17.2mm}\includegraphics[scale=1.0]{rub_slogan.eps}\\
  
  \nohyphens{
   {\bf \LARGE \sffamily {\thtitle}}
  }\\
  
  \large{
   \thauthor
  }\\
  
  \vspace*{35mm}
  \normalsize{
   Seminarausarbeitung\\
   \today\\
   Embedded Security Group - Prof. Dr.-Ing. Christof Paar\\
  }
 \end{minipage}
\end{titlepage}


\newpage\thispagestyle{empty}


% Abstract
\section*{Abstract}
Due to the progress in the developement of quantum computers, the need for pratical post-quantum cryptography is getting increrasingly stronger. Lattice-based cryptography is just one of a few candidates that could replace the public-key algorithms used these days. However, it is the most promising one.

Though lattice-based cryptography has only come up in recent years, a lot of research has been done in that area. Most papers in recent years have focused on efficient implementations for some of the proposed lattice-based algorithms, without considering the possibility of side-channel attacks. Therefore, we summarize a few proposed side-channel attacks and  countermeasures and evaluate their practicability and efficiency.

In that context, we present two masking schemes and their implementation for a ring-LWE encryption scheme. While just one of whom is sound against all first-order DPA attacks, it lacks in efficiency compared to the other one. The second one is much more efficient, but might still be vulnerable to some refined first-order DPA.

Furthermore, we describe a cache attack on the Gaussian sampling of the Bimodal Lattice Signature Scheme (BLISS) by introducing two different sampling algorithms and explain cache attacks on both. Afterwards, we present experimental attack results with a perfect side channel and a real side channel on Intel CPUs.

Finally, two blinding techniques are shown, which can be used for any arbitrary algorithm, but are not proven to be secure against side-channel attacks. However, they do not add too many computations to the existing algorithms, meaning there will not be a big loss in terms of efficiency.
\clearpage

\tableofcontents
\mainmatter

% List of acronyms
\chapter*{Acronyms}
\begin{acronym}
 	\setlength{\itemsep}{0.2em} 
	\acro{BLISS}{Bimodal Lattice Signature Scheme}
 	\acro{DPA}{Differential Power Analysis}
 	\acro{HO-DPA}{Higher Order Differential Power Analysis}
 	\acro{LPR}{Lyubashevsky-Peikert-Regev}
 	\acro{NTT}{Number Theoretic Transform}
 	\acro{PRNG}{Pseudo Random Number Generator}
 	\acro{ring-LWE}{Ring Learning with Errors Problem}
\end{acronym}

%
% Include all your chapters as .tex files, each file contains sections \section{name of section},
% subsections \subsections{...} and so on...
%

\pagenumbering{arabic} 

%
% Sample introduction of your thesis
%

\chapter{Introduction}
Despite the rapid progress in the development of quantum computers and the hereby increasingly urgent need for post-quantum cryptographic algorithms, no such algorithms has yet been standardized \cite{Nist}. Current public-key cryptosystems like RSA, DHKE or even elliptic curve cryptography could easily be broken by a quantum computer, due to Shor's algorithm for prime factorization and discrete logarithms \cite{Shor}. As most of today's digital infrastructure depends (at least partially) on such public-key algorithms, the need for efficient and secure cryptography that can withstand the power of quantum computation is as high as never before.\\
Lattice-based cryptography is the most promising of all attempts in post-quantum cryptography, as its underlying mathematics are already well understood and reasonably efficient implementations of some of the proposed cryptographic schemes are available today. Our paper will give an overview over some selected lattice-based algorithms and their implementation in respect to their resistance to various side-channel attack techniques.\\

\section{Related Work}
This paper summarizes the content of several other papers, that have been published in recent years. Some of them are referred to below and we strongly recommend to take a look at them.\\
Shortly after the \acs{ring-LWE} problem was introduced in \cite{cryptoeprint:2012:230} in 2012, the authors of \cite{maskedRing} laid the groundwork for masked implementations of one \acs{ring-LWE} encryption scheme and refined it in \cite{Reparaz2016} by getting rid of the need for a masked decoder. Just a year after the \acs{ring-LWE} encryption scheme was introduced, the authors of \cite{bliss} proposed the BLISS signature scheme, which is as well based on the \acs{ring-LWE} problem. A possible side-channel attack on a slightly altered version of that signature scheme was then shown in \cite{cryptoeprint:2016:300} in 2016, which might be prevented by the blinding techniques used by the authors of \cite{cryptoeprint:2016:276}.

\section{Structure of this Paper}
In Section 2 we will start with an explanation of our notation and give an overview over the mathematic background needed to understand this paper. This includes introducing the reader to the concept of \textit{(ideal) lattices}, the \textit{\ac{ring-LWE}}, \textit{Discrete Gaussian Distributions}, a \textit{\ac{ring-LWE} Encryption Scheme} and the \textit{BLISS Signature Scheme}. Additionally, we will give a short explanation of the side-channel attack terminology used throughout this paper.\\ 
Section 3 will deal with the ring-LWE encryption scheme and will be split into two parts, starting with the description of a masked implementation of the decryption function, including a masked decoder build upon a masked table lookup. The second part of this section will be an evaluation of the proposed implementation in respect to its soundness to first- and second-order side-channel attacks.\\ 
A different approach to masking of the \ac{ring-LWE} encryption scheme will be presented in Section 4 of our paper, which will as well be split into a description of the proposed scheme and an evaluation. Furthermore, the second masking scheme will be compared to the first one in respect to efficiency and complexity.\\
Section 5 will discuss two different Gaussian samplers for use in the BLISS signature scheme and their vulnerability to the \verb|FLUSH+RELOAD| cache attack. This part will start with a description of two Gaussian samplers proposed by the BLISS authors, namely the cumulative distribution function (CDT sampling) and rejection sampling, followed by the results of a perfect side channel attack on the two sampling algorithms. This will be followed by an short evaluation of the \verb|FLUSH+RELOAD| attacks on an actual BLISS implementation, while running on modern CPUs.\\
Furthermore, in Section 6 we will be presenting two measures used for blinding polynomial multiplication and Gaussian sampling, which might help against the attacks described in Section 5.\\
Finally, Section 7 will summarize the content of our paper shortly and some conclusions will be drawn.
%
% Sample conclusion of your thesis
%

\chapter{Theoretical Background}

\section{Notation}
As we will only work with ideal lattices in our paper, all operations will be done within the ring \(R_q=\mathbb{Z}_q[x]/(f(x))\) with \(f(x)\) being an irreducible polynomial of degree \(n\) and all coefficients being reduced modulo \(q\).\\
Polynomials will be written as \(f(x)\) and vectors will be denoted by bold lower case letters, while matrices will be denoted by bold upper case letters. The entries of a vector \(\textbf{x}\) will be called \(x_i\), with \(i\) specifying the position within the vector starting at 0.\\
The notation for the \(l_p\) norm of a vector \(\textbf{x}\) will be \(\|x\|_p\), only with the exception of the \(l_2\) norm, which will be referred to as \(\|x\|=\sqrt{\sum_{i} x_i^2}\).

\section{Ideal Lattices}
A lattice \(\Lambda\) is discrete subgroup of \(\mathbb{R}^n\) that is defined as a set of \(m \leq n\) linearly independent vectors \(\textbf{b}_1,...,\textbf{b}_m \in \mathbb{R}^n\) and is generated by all linear combinations of those \(\textbf{b}_i\)'s with integer coefficients:
\begin{center}
	\(\Lambda(\textbf{b}_1,...,\textbf{b}_m)=\left \{ \displaystyle \sum_{i=1}^{m} x_i \textbf{b}_i \: \middle | \: x_i \in \mathbb{Z} \right \}\)
\end{center}
The set \(\{\textbf{b}_1,...,\textbf{b}_m\}\) of those vectors is called the basis of that lattice. Such a basis is commonly represented by a matrix \(\textbf{B}=(\textbf{b}_1,...,\textbf{b}_m)\).\\
Furthermore, an ideal lattice is is lattice that corresponds to ideals in a ring \(R_q\). This basically means, that we can deal with polynomials instead of matrices, which makes arithmetics used for cryptographic applications much more efficient. In our paper we will confine ourselves to those ideal lattices, as most of the current work in that area focuses around them. For more on the topic of ideal lattices, see \cite{cryptoeprint:2012:230}.

\section{Learning with Errors Problem}

\section{Discrete Gaussian Distribution}
The discrete Gaussian distribution with mean \(\mu\) and standard deviation \(\sigma\) is denoted as \(\mathcal{N}_\mathbb{Z} (\mu, \sigma^2)\). In this paper we will focus on zero-centered distributions \(\mathcal{N}_\mathbb{Z} (0, \sigma^2)\) with a density function \(\rho_\sigma(x)\) given by:
\begin{center}
	\(\rho_\sigma(x)=\frac{1}{\sqrt{2\pi \sigma^2}}e^{-\frac{x^2}{2\sigma^2}}\)
\end{center}
The discrete Gaussian distribution over \(\mathbb{Z}\) is then defined as \(D_\sigma(x)=\rho_\sigma(x)/\rho_\sigma(\mathbb{Z})\) with \(\rho_\sigma(\mathbb{Z})=\sum_{y=-\infty}^{\infty} \rho_\sigma(y)\). As we will sample whole vectors most of the time, we define the discrete Gaussian distribution over \(\mathbb{Z}^m\) as \(D_\sigma^m(x)=\rho_\sigma(\textbf{x})/\rho_\sigma(\mathbb{Z})^m\) with \(\rho_\sigma(\textbf{x})\) defined as follows:
\begin{center}
	\(\rho_\sigma(\textbf{x})=\frac{1}{\sqrt{2\pi \sigma^2}}e^{-\frac{\|\textbf{x}\|^2}{2\sigma^2}}\)
\end{center}
\section{Cryptographic Algorithms}

\subsection{Ring-LWE Encryption Scheme}

\subsection{BLISS Signature Scheme}
\begin{algorithm}
    \caption{\textsc{BLISS Key Generation}}
    \begin{algorithmic}[1]
        \Ensure{BLISS key pair \((\textbf{A},\textbf{S})\) with public key \(\textbf{A}=(\textbf{a}_1,\textbf{a}_2) \in R^2_{2q}\) and secret key \(\textbf{S}=(\textbf{s}_1,\textbf{s}_2) \in R^2_{2q}\), such that \(\textbf{AS}=\textbf{a}_1 \cdot \textbf{s}_1 + \textbf{a}_2 \cdot \textbf{s}_2 \equiv q\) mod \(2q\)}
        \State{Choose \(\textbf{f},\textbf{g} \in R_{2q}\) uniformly at random with exactly \(d_1\) entries in \(\{\pm 1\}\) and \(d_1\)
entries in \(\{\pm 2\}\)}
		\State{\(\textbf{S}=(\textbf{s}_1,\textbf{s}_2)=(\textbf{f},2\textbf{g}+1)\)}
		\If{\(\textbf{f}\) violates certain conditions (see \cite{bliss})}
			\State{Restart}
		\EndIf
		\State{\(\textbf{a}_q=(2\textbf{g}+1)/\textbf{f}\) mod \(q\) (restart if \textbf{f} is not invertible)}\\
		\Return{\((\textbf{A},\textbf{S})\) with \(\textbf{A}=(2\textbf{a}_q,q-2)\) mod \(2q\)}
    \end{algorithmic}
\end{algorithm}

\begin{algorithm}
    \caption{\textsc{BLISS Signature Algorithm}}
    \begin{algorithmic}[1]
    	\Require{Message \(\mu\), public key \(\textbf{A}=(\textbf{a}_1,q-2)\), secret key \(\textbf{S}=(\textbf{s}_1,\textbf{s}_2)\)}
        \Ensure{Signature \((\textbf{z}_1,\textbf{z}_2^\dagger,\textbf{c}) \in \mathbb{Z}^n_{2q} \times \mathbb{Z}^n_p \times \{0,1\}^n\)}
        \State{\(\textbf{y}_1,\textbf{y}_2 \leftarrow D_\sigma^n\)}
        \State{\(\textbf{u}=\zeta \cdot \textbf{a}_1 \cdot \textbf{y}_1 + \textbf{y}_2\) mod \(2q\)}
        \State{\(\textbf{c}=H(\lfloor \textbf{u} \rceil_d)\) mod \(p\), \(\mu\)}
        \State{Choose a random bit \(b\)}
        \State{\(\textbf{z}_1=\textbf{y}_1+(-1)^b\textbf{s}_1 \cdot \textbf{c}\) mod \(2q\)}
        \State{\(\textbf{z}_2=\textbf{y}_2+(-1)^b\textbf{s}_2 \cdot \textbf{c}\) mod \(2q\)}
        \State{Continue with a probability based on \(\sigma\), \(\|\textbf{Sc}\|\), \(\langle \textbf{z},\textbf{Sc} \rangle\) (see \cite{bliss}), else restart}
        \State{\(\textbf{z}_2^\dagger=(\lfloor \textbf{u} \rceil_d-\lfloor \textbf{u}-\textbf{z}_2 \rceil_d)\) mod \(p\)}\\
        \Return{\((\textbf{z}_1, \textbf{z}_2^\dagger, \textbf{c})\)}
    \end{algorithmic}
\end{algorithm}

\begin{algorithm}
    \caption{\textsc{BLISS Verification Algorithm}}
    \begin{algorithmic}[1]
    	\Require{Message \(\mu\), public key \(\textbf{A}=(\textbf{a}_1,q-2)\), signature \((\textbf{z}_1, \textbf{z}_2^\dagger, \textbf{c})\)}
    	\If{\(\textbf{z}_1,\textbf{z}_2^\dagger\) violate certain conditions (see \cite{bliss})}
    		\State{Reject}
    	\EndIf
    	\If{\(\textbf{c}=H(\lfloor \zeta \cdot \textbf{a}_1 \cdot \textbf{z}_1 + \zeta \cdot q \cdot \textbf{c} \rceil_d + \textbf{z}_2^\dagger\) mod \(p\), \(\mu\))}
    		\State{Accept}
    	\EndIf
    \end{algorithmic}
\end{algorithm}

\section{Side-Channel Attack Terminology}
\chapter{Masking the Ring-LWE Encryption Scheme Using a Masked Decoder}
Since most side-channel attacks focus on the decryption operation, this section will present an attempt to masking the decryption function of the \textit{\ac{LPR} \ac{ring-LWE}} encryption scheme. This masking approach was originally proposed in \cite{maskedRing}, for more details we would refer you to that paper.

\section{Implementation}
We will start by giving the reader an overview of the general setup, before going into more detail about the masked decoding algorithms. We will make strong use of the \textit{\ac{NTT}} in this chapter. We recall, that our notation for polynomials in the \textit{\ac{NTT}} domain is \(\tilde{\textbf{f}}\). The \textit{\ac{NTT}} operation itself will be denoted as \(\textsc{NTT}(.)\), while its inverse operation will be written as \(\textsc{INTT}(.)\). We want to stress, that \(\textsc{NTT}(.)\) and \(\textsc{INTT}(.)\) are linear operations, as we will use this characteristic for our blinding technique.

\subsection{Overview}
This Subsection will cover a concise overview of the blinding technique proposed in \cite{maskedRing}. For the sake of simplicity, the intermediate \(\textbf{m}_{enc}\) will be referred to as \(\textbf{a}\) in the following.

We start by splitting the secret key \(\textbf{s}\) into two shares \(\textbf{s}',\textbf{s}'' \in R_q\) such that \(\textbf{s}=\textbf{s}'+\textbf{s}''\). Therefore, we choose all coefficients of \(\textbf{s}'\) uniformly at random and calculate \(\textbf{s}''=\textbf{s}-\textbf{s}'\). In the \textit{\ac{NTT}} domain it follows that \(\tilde{\textbf{s}}=\tilde{\textbf{s}}'+\tilde{\textbf{s}}''\). Due to the linearity of \(\textsc{INTT}(.)\) and the multiplication, we can compute \(\textbf{a}\) as:
\begin{equation}
	\textbf{a}=\textsc{INTT}(\tilde{\textbf{s}} \cdot \tilde{\textbf{c}}_1+\tilde{\textbf{c}}_2)=\textsc{INTT}(\tilde{\textbf{s}}' \cdot \tilde{\textbf{c}}_1+\tilde{\textbf{c}}_2)+\textsc{INTT}(\tilde{\textbf{s}}'' \cdot \tilde{\textbf{c}}_1)
\end{equation}
This enables us to split the whole equation into two branches, calculating \(\textbf{a}'\) and \(\textbf{a}''\) in the following way:
\begin{equation}
	\textbf{a}'=\textsc{INTT}(\tilde{\textbf{s}}' \cdot \tilde{\textbf{c}}_1+\tilde{\textbf{c}}_2),\:\textbf{a}''=\textsc{INTT}(\tilde{\textbf{s}}'' \cdot \tilde{\textbf{c}}_1)
\end{equation}
Those computations can be done on a arithmetic processor without any protection against side-channel attacks like \textit{\ac{DPA}}, as both branches are totally independent of our secret key \(\textbf{s}\).

However, the \(\textsc{Decode}(a_{i})\) function in the decryption stage of the \textit{\ac{LPR}} scheme is non-linear and cannot easily be split into two parts. For this reason, we will present a masked decoder in the next Subsection, that takes \(\textbf{a}'\) and \(\textbf{a}''\) as inputs to compute two shares \(\textbf{m}'\), \(\textbf{m}''\) of the decoded message \(\textbf{m}\) in a fairly efficient way.

\subsection{Masked Decoder}
This Section briefly describes a probabilistic masked decoder. We recall, that the \(i\)-th element of \(\textit{a}\) is called \(a_i\) and the shares \((a'_i,a''_i)\) of such an element are chosen in a way, that \(a'_i + a''_i = a_i\) (mod \(q\)). To keep it simple, we will refer to an arbitrary \(a_i\) as \(a\), the same follows for its shares.

For our masked decoder, we do not need to know the exact values of \(a'\) and \(a''\) to compute \(\textsc{Decode}(a)\). The following example helps us to lay down some rules for the decoder: Given \((a', a'')\) with \(0 < a' < q/4\) and \(q/4 < a'' < q/2\). Then know that for \(a=a'+a''\) it follows that \(q/4 < a < 3q/4\) and therefore \(\textsc{Decode}(a)=1\). We only need to know the most significant bits of \(a'\) and \(a''\) to determine, which values they are bounded by.
\begin{figure}[H]
	\centering
	\includegraphics[width=\textwidth]{maskedDecoder_1.png}
	\caption{The basic idea of our masked decoder. The circle represents elements in \(\mathbb{Z}_q\). The first case shown allows us to conclude \(\textsc{Decode}(a)=1\), while we cannot make any guesses about the last two ones. \cite{maskedRing}}
	\label{maskedDecoder_1}
\end{figure}
Figure \ref{maskedDecoder_1} shows our example from above on the left. We can use this knowledge to state a total of four rules, the first of whom is taken from our example:
\begin{itemize}
\item \(0 < a' < q/4\), \(q/4 < a'' < q/2\) \(\implies\) \(a \in (q/4,3q/4)\) \(\implies\) \(\textsc{Decode}(a)=1\)
\item \(q/2 < a' < 3q/4\), \(3q/4 < a'' < q\) \(\implies\) \(a \in (q/4,3q/4)\) \(\implies\) \(\textsc{Decode}(a)=1\)
\item \(q/4 < a' < q/2\), \(q/2 < a'' < 3q/4\) \(\implies\) \(a \in (0,q/4) \cup (3q/4,q)\) \(\implies\) \(\textsc{Decode}(a)=0\)
\item \(3q/4 < a' < q\), \(0 < a'' < q/4\) \(\implies\) \(a \in (0,q/4) \cup (3q/4,q)\) \(\implies\) \(\textsc{Decode}(a)=0\)
\end{itemize}
With swapping \(a'\) and \(a''\) in the above rules, one can obtain another four rules. From the rules it follows that we only need to know the quadrant of each \(a'\) an \(a''\) to infer the output of \(\textsc{Decode}(a)\). However, this does not work for all cases, as Figure \ref{maskedDecoder_1} shows. We can actually only apply those rules in half of the possible cases.

So, what happens if our \((a', a'')\) does not match any rule? We simply need to refresh the splitting by computing \(a' = a' + \Delta_1\) and \(a'' = a'' - \Delta_1\) with \(\Delta_i \in \mathbb{Z}_q\). From \((a' + \Delta_1) + (a'' - \Delta_1) = a' + a'' = a\) it follow, that \(a\) stays unchanged by that refresh. Now that we have a fresh pair \((a',a'')\), we can again try to apply our rules from above. This process can be repeated until all shares have been decoded. Note, that a new \(\Delta_i\) should be chosen for each iteration. About half of the \(a',a''\) are decoded per iteration, so that the amount of decoded shares rises exponentially with the number of iterations. The authors of \cite{maskedRing} propose a number of \(N = 16\) iterations for a satisfactory result.
\begin{figure}[H]
	\centering
	\includegraphics[width=0.7\textwidth]{maskedDecoder_2.png}
	\caption{Hardware implementation of the masked decoder. \cite{maskedRing}}
	\label{maskedDecoder_2}
\end{figure}
A possible hardware implementation of such a masked decoder is shown in Figure \ref{maskedDecoder_2}. The refreshing step is depicted on the left, using a different \(\Delta_i\) in each iteration. The quadrant function used in the next step simply takes a refreshed share \(a'\) or \(a''\) as an input and outputs two bits depending on the quadrant that the share belongs to. Next, a masked table is used to check the two bits against the rules we described above. Finally, the masked table function returns two one bit shares of the decoded message \(\textit{m}\). In our implementation the masked takes additional inputs, like a random bit \(r\) and the output of the last iteration \((m_{i-1}',m_{i-1}'')\). For more details on the masked table lookup, we would like to refer you to the paper of Oscar Reparaz et. al. \cite{maskedRing}.

%\subsection{Masked Table Lookup}
%Wenn noch Text fehlt, wird das hier eingefügt

\section{Evaluation}
Starting with efficiency, Reparaz et. al. showed that this implementation is at least 1.9x times better on a Virtex-2 FPGA than an unprotected high-speed elliptic curve scalar multiplier architecture introduced in \cite{Rebeiro2012}.

Furthermore, as both, the \textit{\ac{LPR} \ac{ring-LWE}} encryption scheme and our masked decoder, are probabilistic, there will always be a chance for errors occurring during decoding. The global error rate of decoding rises significantly when using our masked decoder instead of a deterministic decoder. To offset this effect, we can adapt the number of iterations for the masked decoder. While for \(N=3\) iterations the global error rate is about \(49\) times higher than when using a deterministic decoder, \(N=16\) iterations yield a global error rate that is almost identical with the one of a deterministic decoder. Further improvement could be achieved by increasing the number of iterations again, but this would lead to a significantly higher cycle count and thus to a much more inefficient implementation.

For our evaluation in terms of side-channel attack soundness, we assume the attacker knows the details of our implementation and is aware of the rest of the key while guessing a subkey. Our evaluation will follow three steps: First, we perform a first-order key recovery attack with our source of randomness (\textit{\acs{PRNG}}) turned off. This attack will be successful, showing that our setting is correct. In the second step, we will turn on the PRNG and repeat the same attack, but it should not be successful in this case. Finally, we perform a second-order attack to confirm the correctness of the first two steps.
\begin{figure}[H]
	\centering
	\includegraphics[width=0.7\textwidth]{dpa_1.png}
	\caption{\textit{\acs{PRNG}} is turned off. Graph shows the correlation coefficient increasing with the number of traces for the intermediates \(a'_0\) (left) and \(a''_0\) (right). The correct subkey is shown in red, all other guesses in green. \cite{maskedRing}}
	\label{dpa_1}
\end{figure}
For each of those steps, four different points that cover all relevant steps of the algorithm have been tested. The targets are \(a'_0\), \(a''_0\), the first input to the masked decoder and the first output bit. Pearson's correlation coefficient has been used to compare our guesses with real measurements \cite{Brier2004}.

\textbf{\textit{\acs{PRNG}} off:} When the \textit{\ac{PRNG}} is turned off, sharing of \(\textbf{s}\) in \(\textbf{s}'\) and \(\textbf{s}''\) is deterministic. This translates to the masking being turned off. Figure \ref{dpa_1} shows the correlation coefficient evolving with the number of traces. The attack seems to be successful starting at about a hundred traces.
\begin{figure}[H]
	\centering
	\includegraphics[width=0.7\textwidth]{dpa_2.png}
	\caption{Same as Figure \ref{dpa_1}, but with the \textit{\acs{PRNG}} turned on. It is no longer possible to identify the correct subkey within all guesses, meaning that the masking is successful. \cite{maskedRing}}
	\label{dpa_2}
\end{figure}

\textbf{\acs{PRNG} on:} When the \textit{\acs{PRNG}} is turned on, the masking is effective. As we can see in Figure \ref{dpa_2}, the correct subkey can no longer be distinguished from all the other guesses, not even with an enormous amount of traces. This is what an attacker would see when conducting an first-order \textit{\ac{DPA}} on our masking scheme.
\begin{figure}[H]
	\centering
	\includegraphics[width=0.7\textwidth]{dpa_3.png}
	\caption{On the left is the correlation for an increasing number of traces of a first-order attack with masking turned on. On the right we can see a successful second-order attack on our decoding scheme with masking turned on. \cite{maskedRing}}
	\label{dpa_3}
\end{figure}

\textbf{Second-Order Attack:} To confirm that we have used a sufficient number of traces in the first two steps, we perform a second-order attack on our masking scheme. In Figure \ref{dpa_3} we can see, that the second-order attack starts to be successful at around 2000 traces. From this we conclude, that we carried out the first-order attack on the activated masking scheme correctly, as we are already successful with 2000 traces. We stress that an attacker would need a significantly higher number of traces and computations in reality, as we used a pretty friendly setting for our scenario.
\chapter{Additively Homomorphic ring-LWE Masking}
Less than a year after\cite{maskedRing} was published (which has been described in the last Section), Reparaz et. al. presented a follow-up paper \cite{Reparaz2016} introducing a much easier approach for the masking of the \ac{ring-LWE} encryption scheme.

In the following, the approach of said paper will be introduced and evaluated in terms of efficiency and side-channel attack resistance. In our description we again focus on the \ac{LPR} scheme, though the techniques described below are also applicable for other \ac{ring-LWE} encryption schemes.

\section{Implementation}
For \ac{ring-LWE} masking, the authors make use of the fact that the \ac{LPR} encryption scheme is additively homomorphic. Thus for given ciphertexts \((\textbf{c}_1, \textbf{c}_2)\) and \((\textbf{c}'_1, \textbf{c}'_2)\), which are the encryption of two messages \(\textbf{m}\) and \(\textbf{m}'\) with \(m_i, m'_i \in \{0,1\}\) using the same public key \(\textbf{p}\), it follows that \((\textbf{c}_1+\textbf{c}'_1, \textbf{c}_2+\textbf{c}'_2)\) is the encryption of \(\textbf{m} \oplus \textbf{m}'\). As a consequence, the following equation can be written down:
\begin{equation}
	decryption(\textbf{c}_1,\textbf{c}_2) \oplus decryption(\textbf{c}'_1,\textbf{c}'_2) = decryption(\textbf{c}_1 + \textbf{c}'_1,\textbf{c}_2 + \textbf{c}'_2)
\end{equation}
Now, the aim is to make use of the property of additive homomorphism for the proposed masking scheme. This, again, focuses on the decryption function, as this is the part of the encryption scheme, where the secret key is used, which makes it a prime target for attackers.

To randomize the decryption of \((\textbf{c}_1, \textbf{c}_2)\), we need to follow three simple steps:
\begin{enumerate}
\item Generate a random message \(\textbf{m}'\) unknown to the adversary
\item Encrypt \(\textbf{m}'\) to \((\textbf{c}'_1, \textbf{c}'_2)\)
\item Decrypt \((\textbf{c}_1+\textbf{c}'_1, \textbf{c}_2+\textbf{c}'_2)\) to receive \(\textbf{m} \oplus \textbf{m}'\)
\end{enumerate}
The masked message returned by this approach is \((\textbf{m}', \textbf{m} \oplus \textbf{m}')\), such that \(\textbf{m} = (\textbf{m} \oplus \textbf{m}') \oplus \textbf{m}'\).

The advantage of this approach is that no masked decoder is needed. For decoding, an unprotected decoder might be used without leaking any useful information for an attacker.

\section{Evaluation}
As \(\textbf{m}'\) and \((\textbf{c}'_1, \textbf{c}'_2)\) could be precomputed due to the independency of \(\textbf{m}'\), the tradeoff in terms of efficiency seems to be pretty reasonably. Those values can simply be computed, while the processor is in \textit{idle} mode. It is important to stress that fresh values \(\textbf{m}'\) and \((\textbf{c}'_1, \textbf{c}'_2)\) are needed for each encryption. Note that it was not possible to do any precomputations when using the masked decoder described in the previous section.

Furthermore, the complexity of the implementation of this approach is much lower compared to the other approach, which needs a carefully implemented masked decoder. However, the error rate of the new scheme is about a hundred times higher than the \ac{ring-LWE} encryption scheme with masking being turned off. This effect might be compensated by increasing the modulus \(q\) by one bit from 13 to 14 bits or by decreasing the standard deviation \(\sigma\) of the discrete Gaussian distribution, though this would result in a decrease of security.

Although, straightforward first-order \ac{DPA} attacks do not immediately work on our masking approach, more refined ones are still possible. This is due to the fact, that the key itself is not masked. The masking is only making it harder for an attacker to model the power consumption for first-order \ac{DPA}. Using the same technique as before, the authors of \cite{Reparaz2016} showed that while first-order \ac{DPA} attacks on the decryption with the masks being turned off are easy to conduct, the same attack on the scheme with masking being turned on is not possible straight away. More details on this can be found in the corresponding paper.



%
% Sample conclusion of your thesis
%

\chapter{Flush+Reload Cache Attack on Bliss}
\label{bliss}

\section{Gaussian Sampling} %Eventuell zur Theorie

\subsection{CDT Sampling}
Using the cumulative distribution function in the sampler, we build a large table, in which we approximate the probabilities $p_y=\mathbb{P}[x \le y| x \leftarrow D_\sigma ]$ with $\lambda$ Bits of precision. At sampling time, we generate a uniformly random $r \in [0,1)$ and perform a binary search in the table to locate $y \in [-r\sigma, r\sigma]$, so $r \in [p_{y-1}, p_y)$. If restricted to the non-negative part $[o, r\sigma]$, the probabilities are $p^*_y = \mathbb{P}[|x| \le y| x \leftarrow D_\sigma]$, sampling is still $r \in [0,1)$, but $y \in [0, r \sigma]$ is located. 

The binary search in this sampling method can take some time, so one can speed it up by using an additional \textit{guide table I}. This table stores for example 256 entries consisting of intervalls $I[u] = (a_u, b_u), u \in \{0,...255\}$ such that $p^*_{a_{u}} \le u/256$ and $p^*_{b_{u}} \ge (u+1)/256$. At sampling time, the first byte of $r$ is then used to select the corresponding $I[u]$, which leads to a smaller interval to binary search. $r$ is effectively picked byte-by-byte using the guide table approach. Algorithm \ref{algcdt} summarizes the guide table approach.
 \begin{algorithm}
 	\caption{CDT Sampling With Guide Table}
 	\label{algcdt}
 	\begin{algorithmic}[1]
 		\Require{ Big table $T[y]$ containing values $p^*_y$ of the cumulative distribution function of the discrete Gaussian distribution (using only non-negative values), omitting the first byte. Samll table $I$  consisting of the 256 intervals.}
 		\Ensure{Value $y \in [-r\sigma, r\sigma]$, sampled with probability according to $D_\sigma$}
 		\State{pick a random byte $r$}
 		\State{Let $(I_{min}, I_{max}) = (a_r, b_r)$ be the left and right bounds of interval $I[r]$}
 		\If{$I_{max}-I_{min} = 1$}
	 		\State{generate a random sign bit $b \in \{0,1\}$}
	 		\State
	 		\Return {$y=(-1)^bI_{min}$}
	 	\EndIf
	 	\State{Let $i=1$ denote the index of the byte to look at}
	 	\State{Pick a new random byte r}
	 	\While{$1$}
			\State{$I_z = \lfloor \frac{I_{min}+I_{max}}{2}\rfloor$} 
			\If{$r > (i$th byte of $T[I_z])$}
				\State{$I_{min} = I_z$}
			\ElsIf{$r < (i$th byte of $T[I_z])$}
				\State{$I_{max} = I_z$}
			\ElsIf{$I_{max}-I_{min} = 1$}
				\State{generate a random sign bit $b \in \{0,1\}$}
				\State
				\Return{$y=(-1)^bI_{min}$}
				\Else
				\State{increase $i$ by $1$}
				\State{pick new random byte r}
			\EndIf	
		\EndWhile
 	\end{algorithmic}
 \end{algorithm}
\subsection{Rejection Sampling}
Rejection Sampling basically accepts a sampled uniformly random integer $y \in [-r\sigma, r\sigma]$ with probability $p_\sigma(y)/p_\sigma(\mathbb{Z})$. This is done by sampling a uniformly random value $r \in [0,1)$ and accepting the unformly random $y$ if $r \le p_\sigma(y)$. This procedure can be quite expensive, because $p_\sigma(y)$ has to be calculated to a high precision and even then the rejection rate may be quite high.

The authers of the paper which introduces BLISS (QUELLE) proposed a more efficient rejection sampling algorithm, which will be used in the following chapters.%kann man das so schreiben

This algorithm reduces the amount of rejected samples significantly. It begins with sampling a value $x$ according to the binary discrete Gaussian distribution $D_{\sigma 	_{2}}$ with $\sigma _2 = \frac{1}{2 ln 2}$. Uniformly random bits can be used to do this efficiently. $y = Kx + z, z \in \{0,...,K-1\}$ is then uniformly random sampled and $K = \lfloor\frac{\sigma}{\sigma _2} + 1\rfloor$ is distributed according to the targeted discrete Gaussian distribution $D_\sigma$ by rejecting when $b = \exp (−z(z + 2Kx)/(2σ^2)) = 0$ holds.

This last step still requeres some computing because of the exponential value $b$, but the authors provided a more efficiant algorithm for this, too. %algorithmen einfügen


\section{The FLUSH+RELOAD Attack}
The attack used in PAPER HIER EINFÜGEN is the FLUSH+RELOAD cache attack. This attack abuses the fact, that modern CPUs feature multiple different cache levels, for example L1 cache. This is the fastest and smallest cache level and located closest to the cpu core. The next level, L2 cache, is bigger and slower than L1, L3 is even bigger and slower than L2, etc.

If the CPU has to access a memory address, it looks for the corresponding memory block in higher levels first. If the block is found, a \textit{cache hit}, the data is accessed. But in case of a \textit{cache miss}, it starts to look for the memory block on lower cache levels down to the system memory. If the corresponding block is found in lower levels, the processor \textit{evicts} a cache line in the higher memory and places the found memory line there, to speed up future access on the same memory block.

The higher access times on lower cache levels are exploited by cache timing attacks like the FLUSH+RELOAD attack. The attacker has to use the same memory as the victim to apply this kind of attack. He can monitor the state of the cache and use the timing differences to check which memory blocks are cached and which addresses were accessed by the victim. If done correctly, the attacker can deduce the cache lines of the victims table access, which limits the possible chosen values.


The FLUSH+RELOAD attack abuses the x86\_64 instruction \verb|clflush| to evict a memory block from cache, before the victim executes his algorithm. After the victim is done with his memory access, he measures the time to access the memory block again. If the victim accessed the memory block, the block will be in a fast cache level and the access time will be low. If it was not accessed, the CPU has to load the memory block from a lower cache level and the access will be much slower. The attacker then knows, if the flushed memory block was accessed or not.

\subsection{Attacking CDT Sampling}
The CDT sampling algorithm with an interval table $I$ and a table with actual values $T$ is attackable by the attack described above.

\subsection{Attacking Rejection Sampling}
When rejection sampling is used in the BLISS scheme, the side channel has to decide, if there was a table acces in the $ET$ table. The attack exploits the small size of the $ET$ table, which leaks very precise information about the sampling process.

Depending on the bit $i$ of input $x$, $ET[i]$ is accessed, if $x = 0$, no table look-up is performed. If the attacker detects this, he knows $z=0$ is the sampled value in step ALGORITHMUSSTEP in AlgBLA. In this case the attacker can assume $y \in \{0, \pm K, \pm 2K,...\}$ for usable cache access patterns.

So the attacker knows the coefficients $y_i \in \{0, \pm K, \pm 2K,...\}, i \in \{0,...,n-1\}$ of the noise polynomial y. Because anyone can check if $max|\langle s,c \rangle |\le \kappa < K$ with the public parameters, $y_i$ can be determined completely with the signature vector $z$. With $N$ more signatures $(z_j, c_j), j =1,...N$, the attacker can search for  $y_{ji} \in \{0, \pm K, \pm 2K,...\}$ ($y_{ji}$ means the ith coefficient of$y_j$). If the attacker additionally sees, that $z_{ji} = y_{ji}$, he knows $\langle s, c_{ji} \rangle = 0$. Such vector is a \textit{good vector} for use in the attack and $\zeta _k = c_{ji}$ is saved for later (some known $y_{h}$ will be discarded, if they dont satisfy the necessary requirements). With $n$ of these vectors $\xi _k = c_{ji}; 0 \le i \le n-1, 1 \le j \le N, 1\le k \le n$ a matrix $L \ in \{-1,0,1\}^{nxn}$. The column vectors $\xi_k$ satisfy $sL = 0$ (0 is the all-zero vector) and most likely the only dependency of$\xi_k$ is introduced by s, so s is the only kernel vector. There is no need to randomize this process, because the all-zero vector is used.

ALGORITHMUS   

\section{Attacking the Sampling Algorithms with a Perfect Side Channel}
The paper QUELLE provided experimental results for a perfect side-channel attack using the above attack. This requires the attacker to get every cache line of every table look-up while computing $y$ in CDT and rejection sampling algorithms. The victim is assumed to sign random messages and the signatures are collected by the attacker. Cache lines are 64 byte and each element is 8 byte. 
To fulfill this requirements, the authors of QUELLE modified the \textit{reasearch oriented} C++ implementation published by the BLISS authorsQUELLE. NTL was used for LLL reductions and kernel calculations.
\subsection{Perfect Side Channel Attack on CDT Sampling}
\subsection{Perfect Side Channel Attack on Rejection Sampling}
The perfect side channel tells the attacker, if there was an table loo-up in the table $ET$. The attack explained  in section REFHIER can be applied and requires $m = n$ challenges $c_i$ for the kernel calculation, because no randomization is needed. By checking the cache lines of the small part of the table, the attacker can learn if any element has been accessed in $ET$. Only 100 experiments were performed in this case by the authors of PAPERHIER, because for all parameter sets $p_success = 1.0$ holds.

The expected numbers of signatures needed is:
\begin{equation*}
	\mathbb{E}[N] = ((\frac{1}{p_\sigma(\mathbb{Z})}\sum_{x=-r\sigma}^{r\sigma}p_\sigma(xK))*\mathbb{P}[\langle s_1,c \rangle = 0])^{-1}	
\end{equation*}
with $K= \lfloor \frac{\sigma}{\sigma_2}+1\rfloor$ and tail-cut $r\ge 1$. This number is heavily dependent on the value $\sigma$, because $xK$ is more likely sampled for small $\sigma$

\section{Evaluation}
%
% Sample conclusion of your thesis
%

\chapter{Blinding Countermeasures}
Blinding is a countermeasure commonly used to prevent side-channel attacks like \textit{\ac{DPA}} \cite{DBLP:conf/crypto/KocherJJ99}. It is used to add additional randomness to mathematical operations in a way, that the attacker cannot easily draw conclusions from his observations. This Section summarizes two blinding countermeasures presented in \cite{cryptoeprint:2016:276}, which appear to be of special interest for \ac{ring-LWE} cryptosystems. While the first countermeasure will be an approach to blinding of polynomial multiplication within a ring \(R_q\), the second one will be a blinding countermeasure for Gaussian sampling. All those techniques are believed to help against the attacks we described in \ref{bliss}, yet this has not been verified.

\section{Blinding Polynomial Multiplication}
There are two pretty types of blinding for polynomial multiplications, the first of whom is the multiplication of each polynomial with a constant. For two polynomials \(\textbf{f},\textbf{g} \in R_q\) and constants \(a,b \in \mathbb{Z}_q\) the blinding operation and the inverse operation look as follows:
\begin{equation}
    \textbf{h} = a \textbf{f} \cdot b \textbf{g}
\end{equation}
\begin{equation}
    \textbf{f} \cdot \textbf{g} = (ab)^{-1} \textbf{h}
\end{equation}
The second type would be circularly shifting the coefficients in each of the polynomials. As a polynomial can be written as \(\textbf{f} = \sum_{i=0}^{n-1} f_i x^i\) with \(f_i\) being the \(i\)-th coefficient of the polynomial \(\textbf{f}\), a shift by \(j\) positions would be equal to the following computation:
\begin{equation}
    x^j \textbf{f} = \sum_{i=0}^{n-1} f_i x^{i+j} = \sum_{i=0}^{n-1} f_{i-j} x^i
\end{equation}
Both of those blinding operation can be combined within one function, which will be called \(\textsc{PolyBlind}(\textbf{v},s,c)\) from now on. This function works on the coefficient vectors of polynomials of degree \(n\) and is given in Algorithm \ref{alg:PolyBlind}.\newline
\begin{algorithm}
    \caption{\textsc{PolyBlind}}
    \label{alg:PolyBlind}
    \begin{algorithmic}[1]
        \Require{coefficient vector \(\vec{v}\), number of shifts \(s\), constant \(c\)}
        \Ensure{blinded coefficient vector \(\vec{v'}\)}
        \For{\(i=0,...,n-s-1\)}
            \State{\(v_i'=cv_{i+s}\) mod \(q\)}
        \EndFor
        \For{\(i=n-s,...,n-1\)}
            \State{\(v_i'=q - cv_{i+s-n}\) mod \(q\)}
        \EndFor\\
        \Return{\(\vec{v'}\)}
    \end{algorithmic}
\end{algorithm}
The inverse operation can be denoted by \(\textsc{PolyBlind}(\vec{v'},-s,c^{-1})\). Due to the isometries of the ring \(R_q\), the multiplication of two polynomials (here: their coefficient vectors) can be blinded using the \(\textsc{PolyBlind}\) function in the following way:
\begin{center}
    \(\vec{f'} = \textsc{PolyBlind}(\vec{f},r,a)\) with \(r \in_R {0,...,n-1}\) and \(a \in_R \mathbb{Z}_q\)\\
    \(\vec{g'} = \textsc{PolyBlind}(\vec{g},s,b)\) with \(s \in_R {0,...,n-1}\) and \(b \in_R \mathbb{Z}_q\)\\
    \(\vec{h'} = \vec{f'} \cdot \vec{g'}\)\\
    \(\vec{h} = \textsc{PolyBlind}(\vec{h'},-(r+s),(ab)^{-1})\)
\end{center}

\section{Blinding Gaussian Sampling}
As with the blinding of polynomial multiplication in the last subsection, there are two pretty easy ways to blind the coefficient vectors during the process of Gaussian sampling. We will again give a short description of both of them and present a function, that combines both methods.

We define a function \(\textsc{VectorSample}(n,\sigma)\), that samples and returns a vector according to the discrete Gaussian distribution \(\mathcal{N}_\mathbb{Z}^{n} (0, \sigma^2)\). A naive implementation of this function could lead to leakage of information to an attacker using e.g. DPA. This has been done in the cache attack from \cite{cryptoeprint:2016:300} we described in Section \ref{bliss}.

The first approach to blinding would be to randomly shuffle the elements in the coefficient vector. The function \(\textsc{VectorShuffle}(\vec{x})\) is doing exactly that, so that \(\textsc{VectorShuffle}(\textsc{VectorSample}(n,\sigma))\) would increase security to a certain extend. For the second approach we need to take a short detour through probability theory. For two Gaussian distributions \(X=\mathcal{N}_\mathbb{Z}^{n} (\mu_X, \sigma^2_X)\) and \(Y=\mathcal{N}_\mathbb{Z}^{n} (\mu_Y, \sigma^2_Y)\) it holds that their sum is equal to \(X+Y=\mathcal{N}_\mathbb{Z}^{n} (\mu_X+\mu_Y, \sigma^2_X+\sigma^2_Y)\). As we focus on zero-centered distributions, the center does not change for us. For the standard deviation it follows, that \(\sigma_{X+Y}=\sqrt{\sigma^2_x+\sigma^2_Y}\). Algorithm \ref{alg:VectorBlindSample} is one possible way to make use of those characteristics of Gaussian distributions. It also makes use of the \(\textsc{VectorShuffle}(\vec{x})\) function to increase overall security. Another, very similar approach, has been described in \cite{cryptoeprint:2014:254}.

%#############
%evtl noch weiter ausführen, je nachdem wie viele Seiten wir am Ende noch vollkriegen müssen
%#############

\begin{algorithm}
    \caption{\textsc{VectorBlindSample}}
    \label{alg:VectorBlindSample}
    \begin{algorithmic}[1]
        \Require{length of the vector \(n\), number of iterations \(m\), standard deviation \(\sigma\)}
        \Ensure{sampled vector \(\vec{x}\)}
        \State{\(\textbf{x}=\textbf{0}\)}
        \For{\(i=1,...,m\)}
            \State{\(\vec{x} = \vec{x} + \mathcal{N}_\mathbb{Z}^{n} (0, (\frac{1}{\sqrt{m}}\sigma)^2)\)}
            \State{\(\vec{x} = \textsc{VectorShuffle}(\vec{x})\)}
        \EndFor\\
        \Return{\(\vec{x}\)}
    \end{algorithmic}
\end{algorithm}
\chapter{Conclusion}
This chapter gives an overview of the different attacks and countermeasures we described in this paper and summarizes the bottom line of each chapter.

We start by describing a first-order masking scheme for the decryption function of a \ac{ring-LWE} encryption scheme. After introducing the reader to the masking approach, we present one possible implementation of the proposed masked decoder. Finally, we conclude that this scheme is efficient enough to be considered for real world applications and showed that it is indeed sound against first-order \ac{DPA}.

Furthermore, we present another masking approach for the \ac{ring-LWE} scheme, which is not as sound against first-order \ac{DPA} as the previous one, but outperforms it in terms of efficiency.

Then we introduce different implementations of the Gaussian sampler needed in the \ac{BLISS} signature scheme and explain side-channel cache attacks. The general attack structure is followed by practical experiments with perfect side channels and real implementations, revealing exploitable weaknesses in every suggested \ac{BLISS} parameter set.

Finally, we introduce the reader to two blinding countermeasures, namely blinding of polynomial multiplications and blinding of Gaussian sampling. Both of them can be used for any arbitrary lattice-based cryptosystem, but we do not give any guarantees in terms of side-channel security.




% Generate list of figures
\newpage
\listoffigures

% Generate list of tables
\newpage
\listoftables

% Generate list of algorithms
\newpage
\listofalgorithms
\clearpage

% Generate bibliography with bibtex, the bibfile here is "bibliography.bib", use alphanumerical style
\begin{appendix}
 	\bibliography{bibliography}
 	\bibliographystyle{alpha}
\end{appendix}

\end{document} 
