%
% Sample conclusion of your thesis
%

\chapter{Conclusion}
This chapter gives an overview of the different attacks and countermeasures we described in this paper and summarizes the bottom line of each chapter.

We start by describing a first-order masking scheme for the decryption function of a \ac{ring-LWE} cryptoscheme. After introducing the reader to the masking approach, we present one possible implementation of the proposed masked decoder. Finally, we conclude that this scheme is efficient enough to be considered for real world applications and showed that it is indeed sound against first-order \ac{DPA}.

Furthermore, we present another masking approach for the \ac{ring-LWE} scheme, which is not as sound against first-order \ac{DPA} as the previous one, but outperforms it in terms of efficiency.

Then we introduce different implementations of the gaussian sampler needed in the BLISS signature scheme and explain side-channel cache attacks. The general attack structure is followed by pratical experiments with perfect side channels and real implementations, revealing exploitable weaknessses in every suggested BLISS parameter set.

Finally, we introduce the reader to two blinding countermeasures, namely blinding of polynomial multiplications and blinding of Gaussian sampling. Both of them can be used for any arbitrary lattice-based cryptosystem, but we do not give any guarantees in terms of side-channel security.


