%
% Sample introduction of your thesis
%

\chapter{Introduction}
Despite the rapid progress in the development of quantum computers and the hereby increasingly urgent need for post-quantum cryptographic algorithms, no such algorithms has yet been standardized \cite{Nist}. Current public-key cryptosystems like RSA, DHKE or even elliptic curve cryptography could easily be broken by a quantum computer, due to Shor's algorithm for prime factorization and discrete logarothms \cite{Shor}. As most of todays digital infrastructure depends (at least partially) on such public-key algorithms, the need for efficient and secure cryptography that can withstand the power of quantum computation is as high as never before.\\
Lattice-based cryptography is the most promising of all attempts in post-quantum cryptography, as its underlying mathematics are already well understood and reasonably efficient implementations of some of the proposed cryptographic schemes are available today. Our paper will give an overview over some selected lattice-based algorithms and their implementation in respect to their resistance to various side-channel attack techniques.\\

\section{Related Work}
This paper summarizes the content of several other papers, that have been published in recent years. Some of them are referred to below and we strongly recommend to take a look at them.\\
Shortly after the \acs{ring-LWE} problem was introduced in \cite{cryptoeprint:2012:230} in 2012, the authors of \cite{maskedRing} layed the groundwork for masked implementations of one \acs{ring-LWE} encryption scheme and refined it in \cite{Reparaz2016} by getting rid of the need for a masked decoder. Just a year after the \acs{ring-LWE} encryption scheme was introduced, the authors of \cite{bliss} proposed the BLISS signature scheme, which is as well based on the \acs{ring-LWE} problem. A possible side-channel attack on a slightly altered version of that signature scheme was then shown in \cite{cryptoeprint:2016:300} in 2016, which might be prevented by the blidning techniques used by the authors of \cite{cryptoeprint:2016:276}.

\section{Structure of this Paper}
In Section 2 we will start with an explaination of our notation and give an overview over the mathematic background needed to understand this paper. This includes introducing the reader to the concept of \textit{(ideal) lattices}, the \textit{\ac{ring-LWE}}, \textit{Discrete Gaussian Distributions}, a \textit{\ac{ring-LWE} Encryption Scheme} and the \textit{BLISS Signature Scheme}. Additionally, we will give a short explanation of the side-channel attack terminology used throughout this paper.\\ 
Section 3 will deal with the ring-LWE encryption scheme and will be split into two parts, starting with the description of a masked implementation of the decryption function, including a masked decoder build upon a masked table lookup. The second part of this section will be an evaluation of the proposed implementation in respect to its soundness to first- and second-order side-channel attacks.\\ 
A different approach to masking of the \ac{ring-LWE} encryption scheme will be presented in Section 4 of our paper, which will as well be split into a description of the proposed scheme and an evaluation. Furthermore, the second masking scheme will be compared to the first one in respect to efficiency and complexity.\\
Section 5 will discuss the \verb|FLUSH+RELOAD| cache attacks on the Gaussian sampler used in the BLISS signature scheme. This part will start with a description of a perfect side channel attack on two Gaussian sampling algorithms, namely the cumulative distribution function (CDT sampling) and rejection sampling. This will be followed by an evaluation of the \verb|FLUSH+RELOAD| attacks on an actual BLISS implementation, while running on modern CPUs.\\
Furthermore, in Section 6 we will be presenting two measures used for blinding polynomial multiplication and Gaussian sampling, which might help against the attacks described in Section 5.\\
Finally, Section 7 will summarize the content of our paper shortly and some conclusions will be drawn.